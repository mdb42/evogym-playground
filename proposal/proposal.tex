\documentclass[journal,12pt,onecolumn]{IEEEtran}
\usepackage{mathtools,amssymb,amsfonts}
\usepackage{algorithmic}
\usepackage{algorithm}
\usepackage{graphicx}
\usepackage{xcolor}
\usepackage{float}
\usepackage{setspace}
\usepackage{subcaption}
\usepackage[hidelinks]{hyperref}
\usepackage{multirow}

\doublespacing

\hypersetup{
   colorlinks=true,
   linkcolor=blue,
   citecolor=black,
   urlcolor=blue
}

\usepackage{titlesec}
\titlespacing*{\section}{0pt}{12pt plus 4pt minus 2pt}{12pt plus 2pt minus 2pt}
\titlespacing*{\subsection}{0pt}{12pt plus 4pt minus 2pt}{8pt plus 2pt minus 2pt}
\titlespacing*{\subsubsection}{0pt}{12pt plus 4pt minus 2pt}{6pt plus 2pt minus 2pt}

\title{Neuroevolutionary Control for Soft Robots \\ Under Morphological Degradation}
\author{
   \IEEEauthorblockN{Matthew D. Branson} \\
   \IEEEauthorblockA{\textit{Department of Computer Science} \\
   \textit{Missouri State University}\\
   Springfield, MO \\
   branson773@live.missouristate.edu
   }
}

\date{June 30, 2025}

\begin{document}
\maketitle

\section{Introduction}

\subsection{Problem Statement}
Soft robots operating in real-world environments undergo progressive morphological changes from actuator wear, material fatigue, and component degradation. While evolutionary computing excels at discovering novel control strategies, current approaches evaluate fitness under static conditions that don't reflect real-world degradation. This project explores whether neuroevolutionary methods can discover controllers that maintain performance throughout a robot's operational lifetime, not just at peak condition.

\subsection{Context and Background}
EvolutionGym \cite{evogym2021} offers a comprehensive benchmark for evolving soft robots, including joint morphology and control optimization. NEAT \cite{stanley2002evolving} evolves both neural network topology and weights via complexification, while HyperNEAT \cite{stanley2009hypercube} uses compositional pattern-producing networks to generate scalable neural structures with geometric regularities. These neuroevolutionary approaches succeed in discovering effective controllers, but they target peak performance under static conditions. The gap between simulation assumptions and real-world degradation presents a key obstacle for practical deployment.

\section{Proposed Solution}

\subsection{Approach}
Both NEAT and HyperNEAT will be implemented within the EvolutionGym framework to evolve neural network controllers for soft robots. During evolution, robots will experience progressive degradation through usage-based wear models where actuator degradation correlates with cumulative applied forces, activation frequency, and peak stress events. Two degradation mechanisms will be explored: continuous strength decay during episodes and discrete morphological changes between evaluations. Fitness functions will optimize for lifetime performance, incorporating both task achievement and degradation resilience across varied wear patterns.

\subsection{Innovation}
This work introduces explicit degradation modeling into the neuroevolutionary process for soft robotics. Unlike standard approaches that optimize for static peak performance, this method evaluates controllers across changing morphological conditions, potentially discovering conservative locomotion strategies that trade initial performance for long-term robustness. The direct comparison of NEAT versus HyperNEAT under degradation conditions will reveal whether HyperNEAT's geometric biases provide advantages when handling spatially-correlated degradation patterns.

\section{Success Metrics}

\subsection{Evaluation Methods}
The effectiveness of evolved controllers will be assessed using the following metrics and experimental protocols:

\begin{enumerate}
    \item \textbf{Performance Retention Ratio}: The ratio of final to peak performance, measuring robustness to degradation over time.
    \item \textbf{Cumulative Lifetime Reward}: The total task performance accumulated across the robot’s operational lifespan, capturing both efficiency and durability.
    \item \textbf{Performance Variance}: The consistency of task performance under progressive degradation, reflecting controller stability.
    \item \textbf{Baseline Comparisons}: Comparison against (i) standard NEAT/HyperNEAT evolved without degradation modeling and (ii) random controllers to quantify the benefit of degradation-aware evolution.
    \item \textbf{Statistical Significance}: Multiple independent evolutionary runs will be conducted with t-tests to establish statistical significance between degradation-aware and standard approaches.
    \item \textbf{Ablation Studies}: Systematic removal of degradation components (e.g., disabling usage-based decay or episodic morphology shifts) to evaluate their individual contributions to overall robustness.
\end{enumerate}

\subsection{Expected Outcomes}
Controllers evolved with explicit degradation modeling are expected to demonstrate superior performance retention compared to standard approaches, potentially discovering conservative strategies that trade initial speed for long-term reliability. HyperNEAT may show advantages when handling spatially-correlated degradation due to its geometric encoding biases. Key deliverables include working implementations of NEAT and HyperNEAT within EvolutionGym, a modular degradation simulation framework, and comprehensive comparative analysis of how different neuroevolutionary approaches handle morphological degradation.

\bibliographystyle{IEEEtran}
\bibliography{references}

\end{document}