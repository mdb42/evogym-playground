\documentclass[conference]{IEEEtran}
\IEEEoverridecommandlockouts
% The preceding line is only needed to identify funding in the first footnote. 
% If that is unneeded, please comment it out.
\usepackage{cite}
\usepackage{amsmath,amssymb,amsfonts}
\usepackage{algorithmic}
\usepackage{graphicx}
\usepackage{textcomp}
\usepackage{xcolor}
\usepackage[hidelinks]{hyperref}  % Added for clickable links
\def\BibTeX{{\rm B\kern-.05em{\sc i\kern-.025em b}\kern-.08em
    T\kern-.1667em\lower.7ex\hbox{E}\kern-.125emX}}
\begin{document}

\title{Neuroevolutionary Control for Soft Robots: \\ A NEAT Implementation Study}

\author{\IEEEauthorblockN{Matthew D. Branson}
\IEEEauthorblockA{\textit{Department of Computer Science} \\
\textit{Missouri State University}\\
Springfield, MO \\
branson773@live.missouristate.edu}
}

\maketitle

\begin{abstract}
This paper presents a from-scratch implementation of NeuroEvolution of Augmenting Topologies (NEAT) applied to the coevolution of morphology and control in soft robots within the EvolutionGym framework. The study focuses on walker robots, exploring how simultaneous evolution of body structure and neural network controllers can produce effective locomotion strategies.
\end{abstract}

\begin{IEEEkeywords}
NEAT, neuroevolution, soft robotics, coevolution, morphology, neural networks, EvolutionGym
\end{IEEEkeywords}

\section{Introduction}
The design of effective controllers for soft robots presents unique challenges due to their high-dimensional, non-linear dynamics and complex morphological properties. Traditional control approaches often struggle with the computational complexity and adaptability requirements of soft robotic systems. Neuroevolutionary methods, particularly NeuroEvolution of Augmenting Topologies (NEAT), offer promising alternatives by simultaneously optimizing both neural network topology and parameters.

This study implements NEAT from scratch and applies it to the coevolution of morphology and control in soft robots using the EvolutionGym framework \cite{evogym2021}. The focus is on walker robots, examining how the simultaneous evolution of body structure and neural network controllers can discover effective locomotion strategies.

\section{Related Work}
The EvolutionGym benchmark \cite{evogym2021} provides a comprehensive framework for evolving soft robots, supporting both morphological and control optimization. This platform enables researchers to explore the complex interactions between robot structure and behavior in simulated environments.

NEAT \cite{stanley2002evolving} revolutionized neuroevolution by introducing topology evolution through complexification, starting with minimal networks and gradually adding nodes and connections. This approach addresses the competing conventions problem and enables the discovery of appropriate network architectures alongside optimal weights.

HyperNEAT \cite{stanley2009hypercube} extends NEAT's principles by using compositional pattern-producing networks (CPNs) to generate neural network weights based on geometric relationships, showing particular promise for spatially-structured problems.

\section{Methodology}
This section describes the implementation of NEAT and its application to soft robot evolution within the EvolutionGym framework.

\subsection{NEAT Implementation}
A complete NEAT implementation was developed from scratch, including:
\begin{itemize}
    \item Genome representation with node and connection genes
    \item Crossover operators that preserve topological structure
    \item Mutation operators for adding nodes and connections
    \item Speciation mechanism to protect topological innovation
    \item Fitness sharing within species
\end{itemize}

\subsection{Coevolution Framework}
The study focuses on the coevolution of robot morphology and neural network controllers for walker robots. 

\subsection{Experimental Setup}
Experiments were conducted using walker robots in the EvolutionGym environment.

\section{Results}
This section presents the results of applying NEAT to morphology-control coevolution in soft robots.

\section{Conclusion}
This study demonstrates a from-scratch implementation of NEAT applied to the coevolution of soft robot morphology and control. The work provides insights into how neuroevolutionary approaches can discover effective locomotion strategies through simultaneous optimization of body structure and neural network controllers.

\section*{Acknowledgment}

The author would like to thank the professor for guidance throughout this project and acknowledge that the scope was adjusted from the original proposal due to time and resource constraints.

\bibliographystyle{IEEEtran}
\bibliography{references}

\end{document}