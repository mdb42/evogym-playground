\documentclass[journal,12pt,onecolumn]{IEEEtran}
\usepackage{mathtools,amssymb,amsfonts}
\usepackage{algorithmic}
\usepackage{algorithm}
\usepackage{graphicx}
\usepackage{xcolor}
\usepackage{float}
\usepackage{setspace}
\usepackage{subcaption}
\usepackage[hidelinks]{hyperref}
\usepackage{multirow}

\doublespacing

\hypersetup{
   colorlinks=true,
   linkcolor=blue,
   citecolor=black,
   urlcolor=blue
}

\usepackage{titlesec}
\titlespacing*{\section}{0pt}{12pt plus 4pt minus 2pt}{12pt plus 2pt minus 2pt}
\titlespacing*{\subsection}{0pt}{12pt plus 4pt minus 2pt}{8pt plus 2pt minus 2pt}
\titlespacing*{\subsubsection}{0pt}{12pt plus 4pt minus 2pt}{6pt plus 2pt minus 2pt}

\title{Adaptive Neuroevolutionary Control for Soft Robots Under Morphological Degradation}
\author{
   \IEEEauthorblockN{Matthew D. Branson \\ Partner Name} \\
   \IEEEauthorblockA{\textit{Department of Computer Science} \\
   \textit{Missouri State University}\\
   Springfield, MO \\
   branson773@live.missouristate.edu \\
   partner.email@live.missouristate.edu
   }
}

\date{June 27, 2025}

\begin{document}
\maketitle

\begin{abstract}
This paper presents the implementation and analysis of...
\end{abstract}

\begin{IEEEkeywords}
Genetic Algorithms, Binary Encoding, Optimization, Evolutionary Computing
\end{IEEEkeywords}

\section{Introduction}

Evolutionary algorithms (EAs) have shown great promise in solving complex optimization problems across various domains. Recent work in EvolutionGym \cite{evogym2021} has demonstrated the potential for evolutionary approaches in robotics and control tasks. This research builds upon these foundations to explore new methodologies.

\subsection{Motivation}
The motivation behind this work is to explore...

\subsection{Problem Statement}
The primary problem addressed in this study is...

\subsection{Contributions}
The contributions of this paper include...

\subsection{Overview}
This paper is organized as follows...

\section{Related Work}
In this section, we review prior applications of evolutionary algorithms to control tasks and highlight key benchmarks, including those relevant to EvolutionGym.

\subsection{Evolutionary Approaches to Control}
Previous research has demonstrated...

\subsection{Benchmarking and Simulation Environments}
Frameworks such as OpenAI Gym, PyBullet, and EvolutionGym have been used to evaluate...

\subsection{Multi-Objective and Morphological Evolution}
Recent works exploring multi-objective optimization and morphology evolution have shown...

\section{Methodology}
This section details our genome representation, evolutionary process, and implementation within the EvolutionGym environment.

\subsection{Problem Formulation}
Our objective is to evolve controllers/morphologies that maximize performance in...

\subsection{Genome Representation}
We define each individual as a...

\subsection{Evolutionary Operators}
Selection, crossover, and mutation are applied as follows...

\subsection{Fitness Function}
The fitness function rewards agents that...

\subsection{Implementation Details}
All experiments were implemented using...

\section{Experimental Setup}
We describe the selected environments, experimental parameters, and evaluation protocol.

\subsection{Environment Selection}
Tasks were chosen from EvolutionGym to provide coverage across locomotion, manipulation, and mixed-behavior challenges...

\subsection{Evaluation Metrics}
We measured performance using average reward, success rate, and behavior diversity...

\subsection{Hyperparameters}
We used a population size of..., mutation rate of..., over N generations...

\section{Results}
The results of our experiments demonstrate the effectiveness of...

\subsection{Performance Across Environments}
The evolved agents achieved...

\subsection{Comparison to Baselines}
Compared to baseline controllers, our approach...

\subsection{Ablation Study}
To assess the contribution of each component, we...

\section{Discussion}
We discuss the implications, limitations, and potential for future work.

\subsection{Insights from Evolved Behavior}
Analysis of behavior reveals that...

\subsection{Generalization and Overfitting}
Agents were evaluated in perturbed environments to assess...

\subsection{Limitations}
This work is limited by...

\section{Conclusion and Future Work}
In this study, we presented... In future work, we aim to...

\bibliographystyle{IEEEtran}
\bibliography{references}

\end{document}
